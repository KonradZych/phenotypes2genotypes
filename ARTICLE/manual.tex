\documentclass{article}
\usepackage{multicol}
\usepackage{fullpage}
\usepackage[english]{babel}
\usepackage{blindtext}
\addtolength{\oddsidemargin}{-0.4in}
\addtolength{\evensidemargin}{-0.3in}
\addtolength{\textwidth}{0.7in}
\addtolength{\topmargin}{-0.4in}
\addtolength{\textheight}{0.8in}
\title {pheno2geno}
\begin{document}
\maketitle
Your way to save time and money. We earn only citations.

\author{Konrad Zych, Danny Arends, Ritsert C. Jansen}
\newpage
\section{General introduction}
\begin{multicols}{2}
{\noindent}Genetical Genomics (citation) is powerful method, providing world of life sciences with tool to look deep inside the complex relation between genetic information stored by DNA and final outcome of its processing - phenotype. And because this is the core dogma of modern biology, every method helping us to understand it better is of high importance to scientific world. 

{\noindent}Great power often comes for high price, though. To perform GG studies one has to obtain both phenotypic and genotypic data, that are subsequently being matched. This not only elevates costs of experiment, but also introduces number of human errors, e.g. mismatching/mislabeling of arrays.

{\noindent}This brought us back to DNA to phenotype dogma. And we came up with idea to creating genetic map out of gene expression data. This means the same power for less then half of the cost and effort. Procedure is easy enough to be conducted by inexperienced R user and for advanced users we offer variety of extra functionalities to make their analysis fit their needs.
\subsection{R programming language}
R programming language is powerful, yet easy-to-use. There is graphical interface available for Windows, Mac OS and Linux. Every package/function comes with easily accessible help file and, most importantly, R provides user with handfuls of statistical functionalities. To start your adventure with R just go to: http://cran.r-project.org/, select your operating system, install it, and you're ready to enter the world of R.
\subsection{Downloading and installing package}
R packages extends basic possibilities of R enormously. They all have to be structured in the same way, w
\end{multicols}
\newpage
\section{Data}
\begin{multicols}{2}
\blindtext
\subsection{Data files' structure}
\blindtext[2]
\subsection{Population object}
\blindtext[3]
\end{multicols}
\newpage
\section{Basic workflow}
\begin{multicols}{2}
\subsection{Loading data into workflow}
\blindtext
\subsection{Rank product analysis}
\blindtext
\subsection{Preoptimized parameters for most common experimental crosses}
\blindtext
\subsection{Selecting appriopriate markers}
\blindtext
\subsection{Splitting selected markers}
\subsubsection{Parental mean splitting}
\blindtext
\subsubsection{EM splitting}
\blindtext
\subsection{Filtering markers}
\blindtext
\subsection{Cross object}
\blindtext
\subsubsection{Creating cross object}
\blindtext
\subsubsection{Forming linkage groups and \\* ordering markers}
\blindtext
\subsubsection{Augmenting cross object}
\blindtext
\end{multicols}
\newpage
\section{Advanced options/modifications}
\begin{multicols}{2}
\subsection{Using data files with different \\* structure}
\blindtext
\subsection{Rank product analysis}
\blindtext
\subsection{Uncommon types of crosses}
\blindtext
\subsection{Modifying splitting options}
\blindtext
\subsection{Filtering markers}
\blindtext
\subsection{Cross object}
\blindtext
\subsubsection{Forming linkage groups and \\* ordering markers}
\blindtext
\subsubsection{Post-processing of cross object}
\blindtext
\end{multicols}
\newpage
\section{Built-in plotting routines}
\subsection{plotChildrenExpression}
\blindtext[2]
\subsection{plotParentalExpression}
\blindtext[2]
\subsection{plotMapComparison}
\blindtext[2]
\subsection{plotMarkerDistribution}
\blindtext[2]
\newpage
\section{Big datasets}
\begin{multicols}{2}
\subsection{Problematic handling of big data by R}
\blindtext
\subsection{C preprocessing}
\blindtext
\subsection{Other solutions}
\blindtext
\end{multicols}
\newpage
\section{References}
\end{document}